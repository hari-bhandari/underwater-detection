\section{Problem Statement}

The detection of objects underwater is still in its infancy due to the highly dynamic and difficult characteristics of the environments below the water surface. The environment under the water is completely different from the  land environment, where various factors such as underwater visibility, light changes, high turbidity, and varying object scales affect image quality and complicate detection efforts \cite{ancuti2018color,jmse12091606}. Some of the original primary targeting methods like diver observation, sonar imagery, and manual video review have some notable shortfalls. For example, the presence of divers on the target site is limited by time, depth and safety in such a way that their observations do not provide long-term continuous data on the area of interest underwater such observations cover only sea scenes. This gap has driven the exploration of more advanced and reliable solutions. \cite{jmse12091606}
\section{Historical Background}

\subsection{Sonar and Acoustic Imaging}

From the 1980s, different sonar and acoustic imaging technologies have been investigated to tackle problems related to underwater exploration ever since \cite{rajapan2022acoustic}. Sound waves from sonar systems enabled indicating some of notable underwater features like large underwater objects, mapping of underwater surfaces as well as detecting schools of fish. Nevertheless, there were severe drawbacks of the sonar technology such as resolving any small scale or detailed objects – thus low resolution images. This limitation partly affected the level of detail and the type of objects which can be differentiated \cite{kraken_sonar, asp_sonar_resolution}. Also another consideration was the fact that sonar was unable to detect objects at different scales and distances which greatly lowered its effectiveness in the complicated underwater scenes \cite{mdpi_sonar_impact}.

\subsection{Robotic Technologies: ROVs and AUVs}

During this period, robotic technologies, such as ROVs (Remotely Operated Vehicles) and AUVs (Autonomous Underwater Vehicles), underwent rapid and revolutionary advancements. ROVs played a critical role in major explorations, including the discoveries of the RMS Titanic and the German battleship Bismarck in the 1980s \cite{droneblog_rov_history}. Additionally, ROVs were used in the oil and gas industry for deep-water inspection and maintenance, marking the first commercial application of the technology. \cite{hayesparsons_rovs}. This propelled further development in underwater robotics, leading to the introduction of AUVs in the 1990s. AUV activities included long-term oceanographic surveys, military mine detection, and resource extraction. \cite{thedroneu_auvs}.

The modern exploration started with advancements in sonar and underwater robotic systems that originated during the 1980s and 1990s eras enabling sufficient advancements in human activities such as mapping, resource exploration, and navigation within the deep ocean.

\section{Advancements in Underwater Imaging}

Cameras for underwater photography employed in the first decade of the 21st century began to operate at high resolution level. These cameras when used in combination with some preprocessing comprised of colour correction and underwater dehazing, provided clear visual data as they were able to minimize light scattering and distortion of colours. The progress quite improved the functionality of underwater imagery, especially due to the provision of eliminating picture distortion, however, the detection of objects still remained a manual task that was time consuming as well as requiring significant man power for image analysis. Such an automatic procedure was lacking in researchers’ arsenal and could operate efficiently in underwater environments with proper detection and classification of submerged objects in almost real time. This was however not achievable because of the existing methods used in imaging processing not being enough, especially in the underwater environment \cite{ancuti2018color}



\section{Growth of Deep Learning in Object Detection}

The field of object detection has progressed rapidly, largely driven by advancements in Convolutional Neural Networks (CNNs) \cite{chatterjee2018evolution}. The introduction of LeNet-5 \cite{lecun1998gradient} marked the beginning of CNNs' impact on image analysis. Momentum grew with the development of AlexNet \cite{krizhevsky2017imagenet} in 2012, which showcased the potential of deep learning in complex object detection tasks. Subsequent architectures, such as ZFNet \cite{zeiler2014visualizing}, GoogLeNet \cite{szegedy2015going}, and VGGNet \cite{simonyan2015very}, continued to refine and enhance CNN capabilities, laying a strong foundation for underwater object detection applications.

\section{Advances in Preprocessing Techniques}
To deal with these issues, several researchers developed various preprocessing techniques to increase the quality of the underwater image. Color correction algorithms try to compensate for the color distortion, and dehazing techniques reduce the scattering effects to make the visual data clearer by improving the contrast and sharpness of the images. These pre-processing developments have significantly improved the usability of underwater imagery, but they never quite overcame the challenges as real-time detection remained largely manual and laborious. \cite{ancuti2018color}.
